\chapter{Funções Reais}
\label{chap:funções_reais}

\lesson{1}{02 NOV 2022}{Revisão de Funções Reais -- Parte 1}

	A relação entre dois conjuntos $A$ e $B$ é um conjunto de pares ordenados, no qual cada elemento do par ordenado pertence a um dos conjuntos relacionados. Vale saber, não existem restrições (regra) entre os elementos dos conjuntos. Nesse sentido, a função será um caso particular de relação: \textbf{Todo elemento do primeiro conjunto} corresponde a \textbf{um único elemento} do segundo conjunto.

	Uma relação entre dois conjuntos $A$ e $B$ será uma \textbf{função de $A$ em $B$}, se e somente se, para todo $x$ pertencente  a $A$ exista um único $y$ pertencente a $B$, de modo que todo $x$ se relacione com $y$.

	\[
			f(x): A \subset \R \rightarrow B \subset \R
	,\] 

	\[
		x \in A \rightarrow f(x) = y \in B
	.\]		

\begin{figure}[ht]
   \centering
   \incfig[0.7]{conjunto}
   \caption{Relação entre os conjuntos $A$ e $B$ definindo uma função.}
   \label{fig:conjunto}
\end{figure}

	O conjunto $A$ é chamado de \textbf{domínio} da função e o conjunto $B$ é chamado de \textbf{contradomínio} da função. Sendo assim, o contradomínio é todo o conjunto $B$ e os elementos desse conjunto que tem correspondência com o dominínio é denominado de \textbf{imagem}, ou seja, o conjunto imagem é um subconjunto do contradomínio.

	 \section{Funções de uma Variável Real}
	 \label{sec:funções_de_uma_variável_real}
	 
	 Entende-se por uma função $f$ a terna
	 \begin{equation}
		 (A,B,a \mapsto b)
	 \end{equation}
	 em que $A$ e $B$ são dois conjuntos e $a \mapsto b$ uma regra que permite associar a \textit{cada elemento a} de $A$ a \textit{um único elemento b} de $B$. O conjunto $A$ é denominado \textit{domínio de} $f$ e é indicado por $D_f$, assim $A=D_f$. O conjunto $B$ é denominado \textit{contradomínio de} $f$ e é indicado por $CD_f$, assim $B=CD_f$. O único elemento $b$ de $B$ associado ao elemento $a$ de $A$ é indicado por $f(a)$.

	 Uma função de domínio $A$ e contradomínio $B$ é usualmente indicada por $f:A\rightarrow B$. Se $A$ (domnínio) e $B$ (contradomínio) são subconjuntos de $\mathbb{R}$, então $f$ é uma \textit{função de uma variável real a valores reais} tal que $f:A\rightarrow B$.

	 Seja $f:A\rightarrow B$ uma função, seu gráfico será dado pelo conjunto de pares ordenados
	 \begin{equation}
	 G_f = \left\{ \left(x,f\left(x\right)\right) \mid x \in A \right\};
	 \end{equation}
	 assim o gráfico de $f$ é um subconjunto do conjunto de todos os pares ordenados $\left(x,f(x)\right)$ de números reais. Além disso, o gráfico de $f$ pode então se pensado como o lugar geométrico descrito pelo ponto $\left(x,f(x)\right)$ quando $x$ percorre o domínio de $f$.

	 É usual representar uma função $f$ de uma variável real a valores reais e com domínio $A$, simplesmente por \[
	 	y=f(x),x \in A
	 .\] Neste caso, $x$ é a \textit{variável independente} e $y$ a \textit{variável dependente}, ou simplesmente dizer que $y$ é função de $x$.
	 
	 % section funções_de_uma_variável_real (end)

	\section{Classificação de Função Real}
	\label{sec:classificação_de_função_real}
	
	Uma função é denominada \textbf{sobrejetora} se, e somente se, o seu conjunto imagem for igual ao contradomínio.	

	Uma função é denominada \textbf{injetora} se os elementos distintos do domínio tiverem imagens distintas, ou seja, dois elementos não podem ter a mesma imagem.
	
	Uma função será \textbf{bijetora} se ela for, simultaneamente, sobrejetora e injetora.

	Uma função é dita \textbf{par} se apresenta	uma simetria em relação ao eixo vertical.
	\[
		\forall x \in \textbf{Domínio} \rightarrow f(x) = f(-x)
	.\] 
	Uma função é dita \textbf{ímpar} se apresenta uma simetria em relação à origem dos eixos.

	\[
		\forall x \in \textbf{Domínio} \rightarrow f(-x) = -f(x)
	.\]
	Podem existir funções que não são nem par e nem ímpar!

	\begin{figure}[ht]
	   \centering
	   \incfig[0.9]{par_impar}
	   \caption{Definição gráfica de uma função par (a) e uma função ímpar (b).}
	   \label{fig:par_impar}
	\end{figure}
	
	Uma função é dita \textbf{crescente} em um determinado intervalo se os valores de $f(x)$ crescerem quando $x$ crescer neste intervalo: \[
		x_2 > x_1 \iff f(x_2) > f(x_1)
	.\] 
	Uma função é dita \textbf{decrescente} em um determinado intervalo se os valores de $f(x)$ decrescerem quando $x$ crescer neste intervalo: \[
		x_2 > x_1 \iff f(x_2) < f(x_1)
	.\] 

	Uma função é dita \textbf{monótona não decrescente} em um intervalo se os valores de $f(x)$ \textbf{não decrescerem} quando $x$ crescer neste intervalo.\[
		x_2 > x_1 \iff f(x_2) \geq f(x_1)
	.\] Uma função é dita \textbf{monótona não crescente} em um intervalo se os valores de $f(x)$ \textbf{não crescerem} quando $x$ crescer neste intervalo.\[
		x_2 > x_1 \iff f(x_2) \leq f(x_1)
	.\] 
	% section classificação_de_função_real (end)

	\section{Operações com Função Real}
	\label{sec:operações_com_função_real}
	
	Sejam $f$ e $g$ duas funções reais de variável real. Seja $D$ a interseção do domínio de $f$ e do domínio de $g$, então

	\begin{enumerate}
		\item $h(x) = (kf)(x) = kf(x)$, $k$ é real, definida em $D_f$;
		\item $h(x) = (f \pm g)(x) = f(x) \pm g(x)$, definida em $D$;
		\item $h(x) = (f \cdot g)(x) = f(x) \cdot g(x)$, definida em $D$;
		\item $h(x) = \left(\frac{f}{g}\right)(x) = \frac{f(x)}{g(x)}$, definida em $D \cap g(x) \neq 0$;
		\item $f(x) = g(x) \iff I = D_f = D_g$ e $\forall x \in I$, $f(x) = g(x)$.
	\end{enumerate}
	
	% section operações_com_função_real (end)

	\section{Função Composta e Função Inversa}
	\label{sec:função_composta_e_função_inversa}
	
		Uma função é dita \textbf{composta} ou \textbf{função de função} quando é obtida substituindo-se a variável independente por uma função \[
			f \circ g(x) = f(g(x))
		.\] 

		\textbf{Obs.:} \[
			f \circ g \neq g \circ f
		.\] 
		
		Dada uma função $f:A \rightarrow B$, se $f$ é \textbf{bijetora}, então define-se a função inversa $f^{-1}$ como sendo a função de $B \rightarrow A$, tal que $f^{-1}(y) = x$. Os gráficos de $f$ e $f^{-1}$ são simétricos em relação à bissetriz do primeiro quadrante.

\begin{figure}[ht]
   \centering
   \incfig[0.8]{func_inv}
	 \caption{Gráfico da simetria em relação à bissetriz do primeiro quadrante.}
   \label{fig:func_inv}
\end{figure}

	% section função_composta_e_função_inversa (end)
