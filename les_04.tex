\lesson{4}{15 SET 2022}{Funções Reais Elementares -- Parte 2}

\section{Funções Exponencias}

	A função é dita \textbf{exponencial} de base ``$a$'' quando apresenta a seguinte equação:
	\[
		f(x) = ka^{x}, k \in \R^{*}
	.\] 
	Onde a base não pode ser qualquer número, tem que ser $a > 1$ e $a \neq 1$. Se $a = 1$ torna-se uma função constante, pois 1 elevado a qualquer $x$ é sempre 1. O domínio dessa função é $\R$, a imagem é $\R^{*}_{+}$.
\begin{figure}[ht]
    % This in a custom LaTeX template
    \centering
   \incfig[1]{graf-exp}
   \caption{Gráfico da função exponencial}
   \label{fig:graf-exp}
\end{figure}

\begin{description}
	\item [$a > 1$:]  $a^{m} \leq a^{n} \iff m \leq n$
	\item [$0 < a < 1$:] $a^{m} \leq a^{n} \iff m \geq n$
\end{description}

\subsection{Propriedades}
\begin{eqnarray*}
	&&a^{m} = a^{n} \iff m = n \\
	\\
	&&a^{m}\cdot a^{n} = a^{m+n} \\
	\\
	&&\frac{a^{m}}{a^{n}} = a^{m-n} \\
	\\
	&&(a^{m})^{n} = a^{m\cdot n} \\
	\\
	&&a^{-m} = \frac{1}{a^{m}}
\end{eqnarray*}

\section{Funções Logarítmica}

A função é dita \textbf{logarítmica} de base ``$a$'' quando apresenta a seguinte equação:
\[
	f(x) = k \log_{a}(x), k \in \R^{*}
.\]
A base é $a > 0$ e $a \neq 1$. Só existe logaritmo de número positivo, ou seja, $a > 0$. Portanto o domínio é  $\R^{*}_{+}$ e a imagem é todo $\R$.

\begin{figure}[ht]
   \centering
   \incfig[1]{graf-log}
   \caption{Gráfico da função logarítmica}
   \label{fig:graf-log}
\end{figure}

A função logarítmica e a função exponencial são \textbf{função inversas} e terão simetria em relação a bissetriz do primeiro e terceiro quadrante, observe o gráfico \ref{fig:sim-log}.

\begin{figure}[ht]
    % This in a custom LaTeX template
    \centering
   \incfig[1]{sim-log}
   \caption{Simetria em relação a bissetriz do primeiro e terceiro quadrante das funções exponenciais e logarítmicas.}
   \label{fig:sim-log}
\end{figure}

\begin{description}
	\item [Função Crescente] $a > 1$: $\log_a(m) \leq \log_a(n) \iff m \leq n$
	\item [Função Decrescente] $0 < a < 1$: $\log_a(m) \leq \log_a(n) \iff m \geq n$
\end{description}

Um fato a ser notado é quando se tem um $\log$ na base 10. Quando isso ocorre, não se faz necessário escrever a base, assumindo a forma: \[
	\log x
,\] porém quando tem-se o logaritmo na base neperiano, ou seja, sua base é o $e$ o logaritmo é escrito como
 \[
	\ln x
.\] 

\subsection{Propriedades}

\begin{eqnarray*}
	&&\log_a(x) = y \iff a^{y} = x \\
	\\
	&&\log_a(m) = \log_a(n) \iff m = n \\
	\\
	&&\log_a(a) = 1 \\
	\\
	&&\log_a(b^{k}) = k\log_a(b) \\
	\\
	&&\log_{a^{p}}(b) = \frac{1}{p}\log_a(b) \\
	\\
	&&\log_a(x\cdot y) = \log_a(x) + \log_a(y) \\
	\\
	&&\log_a\left(\frac{x}{y}\right) = \log_a(x) - \log_a(y) \\
	\\
	&&\log_a(b) = \frac{\log_c(b)}{\log_c(a)}
\end{eqnarray*}
