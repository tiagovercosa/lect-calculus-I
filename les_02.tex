\lesson{2}{15 SET 2022}{Funções Reais Elementares -- Parte 1}

\section{Função Polinomial}
\label{sec:função_polinomial}
	
	A função é dita \textbf{polinomial} se $f:\R \rightarrow \mathbb{R}$ é dada por
	\begin{equation}
		f(x) = a_nx^{n} + a_{n-1}x^{n-1} + a_{n-2}x^{n-2} + \cdots + a_1x^{1} + a_0,
	\end{equation}
	em que $a_0 \neq 0$, $a_i \in \mathbb{R}$ e $n \in \mathbb{Z}_{+}^{*}$, onde denomina-se função polinomial de grau $n$ $(n \in \mathbb{N})$.
	
	A função polinomial pode ser decomposta em fatores e esses fatores estarão associados com a raíz desse polinômio, podendo ser reescrita como \[
		p(k) = k(x-r_1)(x-r_2) \cdots (x-r_n)
	,\] onde $r_n$ são as  $n$ raízes do polinômio.

	Alguns casos particulares:
	\begin{description}
		\item[Função Constante:] $f(x) = a_0$ 
		\item[Função Linear ou afim:] $f(x) = a_1x + a_0$
		\item[Função Quadrática:] $f(x) = a_2x^2 + a_1x + a_0$
		\item[Função Cúbica:]  $f(x) = a_3x^3 + a_2x^2 + a_1x + a_0$
	\end{description}

	\subsubsection{Função Constante}
	\label{sec:função_constante}
	
	Uma função $y = f(x)$, $x \in A$, dada por $f(x) = k$, $k$ constante, denomina-se \textit{função constante}.

\begin{figure}[ht]
    % This in a custom LaTeX template
    \centering
   \incfig[0.6]{func_const}
   \caption{Função Constante}
   \label{fig:func_const}
\end{figure}

	% subsubsection função_constante (end)

	\subsubsection{Função Linear}
	\label{sec:função_linear}
	
	A função linear é dada por $f:\mathbb{R}\rightarrow \mathbb{R}$ representada por
	\begin{equation}
		f(x) = a_1x + a_0
	\end{equation}
	e tem o gráfico representado na figura abaixo, onde $a_0$ é o \textbf{coeficiente linear} da função, $a_1$ é o \textbf{coeficiente angular} definida por \[
	\tangent \theta = \frac{y_1 - y_0}{x_1 - x_0} = a_1
.\]		
